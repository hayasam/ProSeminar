%=============================================================================
\section{Traceability Methoden, Techniken und Anwendungen}
%=============================================================================
\subsection{Methoden}
\subsubsection{Traceability Schemas \& Metamodelle}
\begin{itemize}

\item
Regeln zur Aufnahme von \textit{traceability} Informationen

\item
Schemas vs. Metamodelle:
\begin{itemize}
\item
\textbf{Schemas:} haupsächlich informale Regeln zur manuellen Aufnahme von \textit{traceability} Information in SE und RE
\item
\textbf{Metamodelle:} Schemas formalisiert als Metamodellen in MDE
\end{itemize}

\end{itemize}

\paragraph{Traceability link types.}
asdf


\subsubsection{Traceability (link) Matrizen}

\begin{center}
\begin{tabular}{|c||c|c|c|c|}
\hline
& A1 & A2 & A3 &  A4 
\\\hline\hline
A1 &  & $\times$  &  &   
\\\hline
A2 &  &  & $\times$  &   
\\\hline
A3 & $\times$ &  &  &   
\\\hline
A4 &  &  &  &  $\times$
\\\hline
\end{tabular}
\end{center}

\begin{itemize}

\item
2D Visualisierung von \textit{traceability links} zwischen Artefakten

\item
Unhandlich für große Anzahlen von Artefakten

\item
Ursprünglich nur Darstellung der Existenz ($\times$) von \textit{traceability links}. Kann aber mit anderen Formen erweitert werden (z.B. $\triangleleft , \triangleright , \bullet ,$ etc.)

\end{itemize}

\subsubsection{Traceability Cross-Refrences}
\begin{center}
\begin{tabular}{|l|l|l|}
\hline
A1 & Lorem ipsum ... & \begin{minipage}[t]{0.1\textwidth}
$\triangleleft$ A2
\\$\triangleright$ A3
\\...\\
\end{minipage} 
\\\hline
A2 & Lorem ipsum ... & \begin{minipage}[t]{0.1\textwidth}
$\triangleleft$ A3
\\$\triangleright$ A3
\\...\\
\end{minipage} 
\\\hline 
A3 & Lorem ipsum ... & \begin{minipage}[t]{0.1\textwidth}
$\triangleleft$ A1
\\$\triangleright$ A2
\\...\\
\end{minipage} 
\\\hline
\end{tabular}
($\triangleleft \approx$ in, $\triangleright \approx$ out)
\end{center}
\begin{itemize}

\item
tabellarische Darstellung von \textit{traceability links} als Artefakt-Metadaten

\end{itemize}

\subsubsection{Traceability Graphen}
\begin{itemize}

\item
intuitive graphische Notation
\begin{itemize}
\item Knoten $\approx$ Artefakte
\item Kanten $\approx$ \textit{traceability links}
\end{itemize}

\item
üblicherweise definiert in Metamodellen


\end{itemize}

%=============================================================================
\subsection{Techniken}
%=============================================================================
\subsection{Anwendungen}
%=============================================================================