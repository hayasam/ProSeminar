\section{Vergleich}

\subsection{Ansätze nach Entwicklungsphasen}
\begin{center}
\begin{tabular}{l|l|l}
Pre-Model
& 
Intra-Model
& 
Post-Model 
\\\hline
\begin{minipage}[t]{0.3\textwidth}
{\tiny
\textit{Architecture frameworks}
\\\textbf{Material \ref{resrc:3}}
\newline
Beschreibt einen Ansatz wie \textit{Traceability} innerhalb eines wiederverwendbaren Architecture Frameworks definiert werden kann. AF beschreiben wie \textit{Traceability} zu implementieren ist.
\\
}
\end{minipage}
&\begin{minipage}[t]{0.3\textwidth}
{\tiny
\textit{Dynamic hierarchical mega models}
\\\textbf{Material \ref{resrc:2}}
\newline
Beschreibt einen Ansatz wie \textit{Traceability} während eines Entwicklungsprozesses konsistent und korrekt erhalten bleibt.
}
\end{minipage}
&\begin{minipage}[t]{0.3\textwidth}
{\tiny
\textit{Linguistic/Semantic architectures}
\\\textbf{Material \ref{resrc:4}}
\newline
Beschreibt einen Ansatz wie \textit{Traceability} während, aber auch nach eine Entwicklungsprozess erhalten/rekonstruiert werden kann.
}
\end{minipage}
\end{tabular}
\end{center}