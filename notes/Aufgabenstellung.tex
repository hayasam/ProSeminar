\section{Aufgabenstellung}

\subsection{Materialien}
\begin{enumerate}

\item
\label{resrc:1}
S. Winkler and J. von Pilgrim. A survey of traceability in requirements engineer- ing and model-driven development. Software and System Modeling, 9(4):529– 565, 2010. 

\item
\label{resrc:2}
A. Seibel, S. Neumann, and H. Giese. Dynamic hierarchical mega models: com- prehensive traceability and its efficient maintenance. Software \& Systems Mod- eling, 9(4):493–528, 2010.

\item
\label{resrc:3}
R. Hilliard, I. Malavolta, H. Muccini, and P. Pelliccione. Realizing Architecture Frameworks Through Megamodelling Techniques. In Proc. of ASE’10, pages 305–308. ACM, 2010.

\item
\label{resrc:4}
Ralf Lämmel and Andrei Varanovich. Interpretation of Linguistic Architectur. Unpublished, 2014

\item
\label{resrc:5}
Thomas Vogel, Andreas Seibel, and Holger Giese. The Role of Models and Megamodels at Runtime.

\item
\label{resrc:6}
Jean-Marie Favre, Ralf Lämmel, and Andrei Varanovich. Modeling the Linguistic Architecture of Software Products.

\item
\label{resrc:7}
IEEE Std 610.12-1990.

\end{enumerate}

\subsection{Fragen \& Aufgaben}
\begin{enumerate}

\item
Was ist \textit{traceability} in MDE und Softwararchitektur? (Insbesondere Material \ref{resrc:1})

\item
Welche Methoden, Techniken und Anwendungen gibt es? 

\item
Wie kommt \textit{traceability} in Megamodellierungsansätzen vor? (Materialien \ref{resrc:2}, \ref{resrc:3} und \ref{resrc:4})

\item 
Vergleiche die Ansätze in Materialien \ref{resrc:2}, \ref{resrc:3} und \ref{resrc:4}.

\end{enumerate}