\section{Ansätze in den gegeben Materialien}

\paragraph{Materiaien}
\begin{itemize}
\item
\textbf{Material \ref{resrc:2}}
\item
\textbf{Material \ref{resrc:3}}
\item
\textbf{Material \ref{resrc:4}}
\end{itemize}



%======================================================================

\subsection{Architecture frameworks (M. \ref{resrc:3})}
\begin{itemize}


\item
Erstellen von \textit{Architecture frameworks} als wiederverwendbare Megamodelle mittels \textit{MEGAF} bestehend aus:
\begin{itemize}
\item \textit{Viewpoints},
\item \textit{System Concerns},
\item \textit{Model Kinds} (Metamodelle),
\item \textit{Stakeholders},
\item und \textit{Correspondence Rules}
\end{itemize}

\item
\textit{Architecture frameworks} beschreiben wie eine konkrete Architektur zur modelieren ist


\item
\textbf{Kein expliziter Ansatz zu \textit{Traceability}}

\item
Mittels \textit{Model Kinds} und \textit{correspondence Rulse} können konkrete \textit{Traceability}-Modelle wiederverwendbar spezifiziert werden

\end{itemize}

%======================================================================

\subsection{Dynamic hierarchical mega models (M. \ref{resrc:2})}
\begin{itemize}

\item 

\end{itemize}

%======================================================================

\subsection{Linguistic architectures (M. \ref{resrc:4})}
\begin{itemize}

\item
MDE-Ansatz: Automatisierte \textit{Traceability}-Analyse.

\item
Materialspezifischer Ansatz: \textit{''Ersetze UML durch DSL''}

\item
(Mega-)Modelle werden einer DSL (\textit{MegaL}) beschrieben.
\begin{itemize}
\item
Megamodelle beschreiben hier den \textbf{Ist}-Zustand eines Systems (im Gegensatz zur üblichen MDE-Praxis, den Zustand eines zukünftigen Systems zu beschreiben)
\item
Metamodelle/Schemas werden häufig durch (Programmier-)Sprachen ersetzt.
\item
Einheiten in \textit{MegaL} werden durch URIs auf physisch existierende Artefakte abgebildet. (URIs bilden bis zur Inhaltsebene von Dateien ab, z.B. kann eine URI auf eine Metode innerhalb einer Klasse innerhalb einer Datei zeigen.)

\item
Relationen in \textit{MegaL} werden durch designierte Programme (\textit{''tools''}) evaluiert.
\end{itemize}

\item
\textit{Traceability Links} entstehen als Nebenprodukt von Relationsevaluierungen.

\item
\textit{Traceability Links} als 1-zu-1-Abbildung einer Regel-URI auf eine Ressource-URI (wobei die angegbene Regel ebenfalls als physikalische Ressource existiert)

\item
Bensoders interessant zur automatischen Spurgewinnung in Generierungs- und Kompilierungsprozessen, die auf diskreten semantischen Regeln basieren und gewisse Annahmen als erfüllt angesehen werden können.  
\newline
\textbf{Beispiel:}
Eine Java \texttt{.class}-Datei ist das Produkt der Kompilierung der korrespondierenden \texttt{.java}-Datei. Nach erfolgreicher Kompilierung kann angenommen werden, dass die \texttt{.class}- und die \texttt{.java}-Datei Element der (\textit{elementOf}-Relation) Java-Sprache sind. Gleichzeitig kann ebenfalls angenomen werden, dass die \texttt{.class}-Datei konform zur (\textit{conformsTo}-Relation) \texttt{.java}-Datei ist, dass die \texttt{.java}-Datei konform zur Java-Grammatik ist und dass die enthaltene Klasse und ihre Methoden konform zu den entsprechenden Grammatikregeln sind.  
\newline
\textbf{Anmerkung:}
\textbf{Obiges Beispiel ist trivial und beschreibt lediglich den Java- Kompilierungsprozess, der den meisten Softwareentwicklern vertraut sein sollte. Ist man allerdings in der Lage kontextspezifischere Regeln zu formulieren und zu beschreiben wie diese auf die Java-Regeln abgebildet werden, ist die Informationsgewinnung ungleich höher einzuschätzen.}

\end{itemize}

%======================================================================

\subsection{Megamodels at Runtime (M. \ref{resrc:5})}
\subsubsection{Runtime Model}
\begin{itemize}

\item
Runtime $\approx$ Execution-time

\item
Phase im Software Lebenszyklus

\item
Instanzen von Metamodellen (diese sind wiederum Instanzen von Meta-Metamodellen, etc.)

\end{itemize}

\subsubsection{Runtime Model Kategorien}
\begin{itemize}

\item
\textit{Implementation Models} [Impl.Models]
\item
\textit{Configuration \& Architectural Models} [Conf+Arch.Models]
\item
\textit{Context \& Resource Models} [Cxt+Rsrc.Models]
\item
\textit{Configuration Space \& Variability Models} [ConfSpace+Var Models]
\item
\textit{Rules, Strategies, Constraints, Requirements and Goals} [RSCRGs]

\end{itemize}