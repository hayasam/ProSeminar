\section{Introduction}\label{sec:Introduction}
Software and computer systems tend to become bigger and more complex to meet higher expectations.
At the same time the complexity of engineering processes creating and maintaining such systems increases proportionally.
One way of taming this rising complexity is the concept of \textit{traceability} which is commonly interpreted as the system ability \textit{''... to describe and follow the life of software artifacts ...''} \cite{ScopedTraceability}, but it is not yet part of an undergraduate curriculum in computer science. 
Neither is \textit{mega-modeling} due to its high level of abstraction, as it deals with models using other models as their elements. 
We argue that it is reasonable for undergraduates to get familiar with both concepts regarding the fact that an achieved degree qualifies to work with even large systems.

In this paper we will outline the basics of traceability and mega-modeling on an undergraduate level, and show how the combination of both can be beneficial to engineers in terms of \textit{system understanding} and \textit{process management automation} so mega-modeling allows one to use model driven engineering (MDE) techniques and technologies.

\subsection{Contributions of this paper}
\subsubsection{Contributions}
\subsubsection{Non-Contributions}

\subsection{Road-map}\label{subsec:Introduction.Road-map}
§\ref{sec:BasicsOfTraceability}
introduces the concept in terms of terminology and motivation of traceability.
§\ref{sec:BasicsOfMegamodeling}
introduces the concept of mega-models.
§\ref{sec:TraceabilityObjectivesRevisited}
revisits traceability objectives and introduces a higher categorization.
§\ref{sec:TraceabilityMegamodels}
summarizes three approaches on traceability, mega-models and traceability mega-models.
§\ref{sec:Conclusion}
concludes the paper.


